
\chapter{Applications}
\label{ch:apps}


So far, the presented Zone is mostly theoretical. Around and crossing hooks have a precisely defined behavior, but beside Zone local storage, no feature answers the motivations for the Zone. I show here how to combine the Zone properties to build solutions to the initial problems.

\section{Asynchronous Error Handling}

Asynchronous error handling exists in two forms, contextual and systematic.

\paragraph{The contextual error handling} consists in using the Zone as a standard try-catch block. The figure \ref{fig:sas-tc} illustrates the parallel between synchronous and contextual error handling.
\begin{figure}[H]
\centering
\makebox[0pt][c]{
\begin{minipage}[l]{0.6\linewidth}
\centering
\begin{lstlisting}
Future<String> future;^^J
^^J
// begin of async try-catch block^^J
(new ErrorZone()).run(() -> {^^J
^^J
\ \ future = asyncCall();^^J
^^J
\ \ // (1)^^J
\ \ String message = future.get();^^J
\ \ System.out.println(message);^^J
});^^J
// end of async try-catch block^^J
^^J
// (2)^^J
String message = future.get();^^J
System.out.println(message);^^J
\end{lstlisting}
\end{minipage}


\begin{minipage}[c]{0pt}
\begin{center}
\rule{0.4mm}{0.3\textheight}
\end{center}
\end{minipage}

\begin{minipage}[r]{0.6\linewidth}
\centering
\begin{lstlisting}
String future;^^J
^^J
// begin of sync try-catch bloc^^J
try {^^J
^^J
\ \ message = syncCall();^^J
^^J
\ \ // (1)^^J
\ \ System.out.println(message);^^J
^^J
} catch (Error e) { ... }^^J
// end of sync try-catch block^^J
^^J
// (2)^^J
^^J
System.out.println(message);^^J
\end{lstlisting}
\end{minipage}
}
\caption{Asynchronous and synchronous try-catch block}
\label{fig:sas-tc}
\end{figure}

In this example, \lstinline{future = asyncCall()} starts an asynchronous task and returns the \lstinline{Future} result of that task. For this exercise, the task will throw an error. Two cases are considered.

\begin{enumerate}
\item Access \lstinline{message} inside the \lstinline{ErrorZone}.\\
The error is not handled by the \lstinline{ErrorZone}. Nothing gets printed, since \lstinline{future.get()} synchronously throws its error inside the \lstinline{ErrorZone}.
\item Access \lstinline{message} outside the \lstinline{ErrorZone}.\\
Upon \lstinline{future.get()}, the result, zoned inside \lstinline{ErrorZone} must cross out the Zone. The error gets handled and a fall-back value is provided.
\end{enumerate}

The \lstinline{ErrorZone} simply has to implement the crossing hook:

\begin{lstlisting}
/*
 * Checks if crossing token 'token' is an error token.
 * If yes, handle the error.
 */
public Token crossOut(Token token) {
  if (token.isErrorToken()) {
    Token fallback;
    // error handling ...
    return fallback;
  } else {
    // do nothing
    return token;
  }
}
\end{lstlisting}

The crossing resolution algorithm then accomplishes the work of selecting correct hooks according to the calling context.



\paragraph{Systematic error handling} is an automation to uniformly wrap asynchronous code \emph{inside} a try-catch block. This can be compared to Java's \lstinline{Thread.UncaughtExceptionHandler} with the differences:

\begin{itemize}
\item The code catches the errors \emph{before} some third-party execution service (a thread pool for example), preventing the execution to reach the \lstinline{Thread.UncaughtExceptionHandler}.
\item The setting is not global but local to the zoned portion of the code. With \lstinline{Thread.UncaughtExceptionHandler}, one has to carefully set and unset the value.
\item Many different default exception handler can be easily emitted by the same thread.
\end{itemize}

Such behavior is provided by a Zone implementing the asynchronous hook:

\begin{lstlisting}

ZonedTask asyncHook(ZonedTask task) {
  return (input) -> {
    ZonedToken result = task.apply(input);
    if (result.isErrorToken()) {
      ZonedToken fallback;
      //handle logic, compute fallbackValue
      return fallback;
    } else {
      return result;
    }
  };
}
\end{lstlisting}

Since around hooks completely apply before crossing out the Zone, cross-out based error handling does not interfere with this code.

\section{One Zone Per Asynchronous Task}

This is not a direct solution to one problem of chapter \ref{ch:motiv}, but an intermediate tool to build other features. The idea is to use an asynchronous hook to bind submitted task inside a new Zone.

\begin{lstlisting}
ZonedTask aroundSubmit(ZonedTask task) {
  // creates a new instance of a custom Zone extension
  Zone newZone = new MyZone();
  ZonedTask newTask = newZone.bind(task);
  return newTask;
}
\end{lstlisting}

\lstinline{MyZone} can be any Zone implementation, with the single constraint that it should override the internal hook application mechanism to do nothing. Otherwise, \lstinline{newTask} may trigger unexpected internal hooks, inherited by enclosing Zones. The problem is that any \emph{internal} hooks will be also applied where \emph{asynchronous} hooks are.

\section{Long Stack Traces}

Implementation of long stack traces use one Zone per asynchronous task to set a Zone value containing the call stack trace of the asynchronous task. With contextual error handling, it simply defines an asynchronous hook to run submitted code inside a new Zone that:
\begin{itemize}
\item Stores as Zone local value the stack trace at the asynchronous submission.
\item Wraps internal code inside a try-catch block to catch any error, extend it with the stored stack trace and re-throw it.
\end{itemize}

Long stack traces are also possible with systematic error handling, but require heavier mechanisms to ensure that the stack trace is always updated before being caught by the \lstinline{catch} clauses of the error handling.

\section{Asynchronous Dependency Tracing}

Chapter \ref{ch:inpractice} discusses in details asynchronous dependency tracking as a concrete experiment. The principle is however simple. Use one Zone per asynchronous task with IDs and implement a crossing hook to store the transitions from one Zone to another one to some data structure of your choice. One Zone per task creates a one-to-one correspondence between asynchronous tasks and IDs, the crossing hook detects and stores execution dependencies between two IDs.
