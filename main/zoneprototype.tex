
\chapter{Zone Prototype}

Realizing a Zone prototype is the first (and crucial) development step. It allows both to understand concepts, use cases, .. , and grab main challenges, needs and difficulties to reach objectives.

\section{Representation, Integration}

No discussion about features, operation before tackling representation and integration concerns.

\begin{itemize}
\item We want the integration to be as transparent to the user as possible. Do not add tones of boilerplate
\item Threading model : Java does not allow context switch. One thread will never preemptively switch task execution. (Every switch is controlled by the programmer).
\item profit from threading model : representation. Sure that one task is executed completely by the thread. One static context field assigned per thread could be updated at beginning and end of thread's task.
\item problem from threading model : integration. Since Thread cannot control the task it executes, one framework can make a thread executing multiple tasks with different contexts (typical thread pool). Hence the integration take place on top (visible) of the framework, not below (invisible).
\end{itemize}

\section{Features}

With representation and integration solution, the Zone is ready to support concrete functionalities.

\begin{itemize}
\item \zonejs and \zonedrt analysis shows up : Zone values, around hook api, hooks inheritance, asynchronous error handling, long stack traces, zone specification through $\lambda$s
\item Zone values : immutability for thread safety
\item Error handler : synchronous and asynchronous binding
\item Around hooks : distinguish synchronous and asynchronous
\item Long call stack : its own small world
\end{itemize}

\section{Limits}

The long call stack example shows weakness of a naive implementation when needing changes. This comes with even more problematic limitations.

\begin{itemize}
\item Support more than simple Runnables : for now we did not discuss about the kind of task that were executed inside the Zone. Most simple one is Runnable : stand-alone, no expected outcome. Extend to general concept of task.
\item Asynchronous error handling : contextual and multiplicity ambiguity. (maybe multiplicity later, speaking of tokens)
\item LCS Many complications for a 'single' feature.
\item What about ASD ? - redundancy
\item finally if one did just care about zone values and don't want to carry every single feature with ? All this call to a general composable abstraction that provides essential primitives to compose and implement specific API.
\end{itemize}