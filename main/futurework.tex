
\chapter{Future Work}

By its innovating capability to handle uniformly (and rather intuitively) synchronous and asynchronous transition, the Zone provides a fertile ground for further developments.

\section{Implementation}

\begin{itemize}
\item Scope projection
\item Zone stack manipulation (now rely on ad-hoc convenient hacks)
\item Buildable API, an extensible DLS to build Zones
\end{itemize}

\section{Theory}

\begin{itemize}
\item Representation on Petri nets is useful for understanding but not well formalized. A strict theorization will allow to relate and compare and even associate the Zone to existing petri nets extensions (as versatile boxes (?) or colored Petri net)
\end{itemize}

\section{Practice}

\begin{itemize}
\item The zone still is a complex abstraction, especially realizing an integration on an execution framework is hard. A good improvement is to collect experience data and compile it as best practices or trouble shooting doc or even rework some aspect of the model that would appear not well defined.
\item performance analysis. The Zone adds a lot of wrapping overhead, determine how it can be usable in production environment.
\end{itemize}
