
\chapter{Conclusion}
\label{ch:conclusion}

We have seen that the Zone for Java brings you innovative building tools. The integration work may be a little complex (as showed the \vertx\ case study), but once this is correctly done, one can almost transparently rely on the Zone as uniform context. In summary, the Zone needs:
\begin{itemize}
\item Explicit binding to asynchronous tasks submitted to the framework.
\end{itemize}

In return, it brings:
\begin{itemize}
\item Persistent context over asynchronous transitions.
\item Modeled execution flow between Zones
\item Flexible programmatic hooks on:
  \begin{itemize}
  \item Zone binding and execution.
  \item Asynchronous submission and execution.
  \item Execution flow across different Zones.
  \end{itemize}
\item Extensible and reusable structure.
\end{itemize}

To close this work, I'll show three ways of research and development on the Zone and a small analysis of how the original problematic evolued through the research.

\section{Development}

Bringing more and easier binding primitives would be very interesting. Even though the binding principle is detailed, the implementation is still not general. This is mainly due to the large amount of standard interfaces present in the \lstinline{java.util.function} package that represent the Zone's task abstraction. Actually, the Zone only handles \lstinline{BiFunction} and requires additional wrap-steps to bind another interface, as \lstinline{Runnable} for example.

When using and probing the Zone, I oftenly encountered the need to ignore parts of the Zone stack. The union-inheritance

The scope projection is one solution to theorize and specify the Zone stack manipulation


\section{Theory}

\begin{itemize}
\item Representation on Petri nets is useful for understanding but not well formalized. A strict theorization will allow to relate and compare and even associate the Zone to existing petri nets extensions (as versatile boxes (?) or colored Petri net)
\end{itemize}

\section{Practice}

\begin{itemize}
\item The zone still is a complex abstraction, especially realizing an integration on an execution framework is hard. A good improvement is to collect experience data and compile it as best practices or trouble shooting doc or even rework some aspect of the model that would appear not well defined.
\item performance analysis. The Zone adds a lot of wrapping overhead, determine how it can be usable in production environment.
\end{itemize}


\section{Asynchronous problematic}

Discuss evolution of point of interest through the research.

\begin{itemize}
\item First : how to implement an asynchronous context with some hooks
\item Final : the real need is an uniform and general behavior on synchronous and asynchronous transition on which one can rely to implement contextual behavior.
\end{itemize}