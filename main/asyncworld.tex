
\chapter{Asynchronous Approaches}
\label{ch:asyncworld}

I don't mean to give here a definition of ``asynchronous''.
This chapter specifies the use of ``asynchronous'' in the current work. Actually, ``asynchronous'' covers different meanings, making it confusing out of context.
Some may think ``asynchronous'' describes a specific method call, with potential callback code that gets executed after completion. Here, I use it in a more general way, to describe any code that does not run \emph{in sequence} of the code that submitted it. It may run later if there is only one execution unit (single-threaded) or in parallel, potentially later if there is two or more execution unit (mulit-threaded).

% \begin{itemize}
% \item Thread : OS preemption or VM preemption based on locking mechanisms.
% \item Thread pools : popular Java framework to optimize threaded programs.
% \item Callbacks : non-preemptive concurrency. Manual execution unit yield using callbacks (js, dart)
% \item Callbacks with thread pools : scalable event loop (\vertx)
% \item Promises or CompletableFutures (in Java) : late-binding callbacks
% \end{itemize}

\section{Threads}

\section{Thread Pools}

\section{Callbacks}

\section{Callbacks on Thread Pools}

\section{Promises}
aka completable future in Java.


In Java, unlike dart and javascript there is no unique imposed asynchronous model but lots of frameworks built on top of natives threads.

Hence, the Zone developed in this project aim to integrate with any approach. The tool must stay consistent regardless of the asynchronous framework. Single threaded, multi threaded, callbacks, futures or promises