
\chapter{Zone Model}

Similarities among limitations of prototype implementation gives intuitions that a unique solution exists. Thus the present model aims for general and composable features, allowing to build specific behaviours.

\section{Application Needs}

What the model need to fill following requirements.

\begin{itemize}
\item include input and output to the zoned code
\item generals hooks that allow error handling implementation as well as on-the-fly Zone definition.
\item resolution of the contextual ambiguity in asynchronous error handling
\end{itemize}

Last elements introduce need of uniform behaviour between synchronous and asynchronous execution.
That is the real point of exec context.

\section{Crossing Principle}

This approach brings the needed uniformity. Opposed to inherited error handler binding in prototype Zone.

\begin{itemize}
\item Tasks bound to Zones produce zoned results.
\item To use a zoned result, one has to extract it in the current Zone, making its content cross Zone boundaries to current Zone and triggering hooks of crossed zones.
\item If results may contain error as well, crossing hooks can implement error handling.
\end{itemize}

\section{Zone on Petri Nets}

PN are very well suited to schematize asynchronous systems. I will use it to propose a theorical view of the Zone.

\begin{itemize}
\item Zone is represented as a set of places, sets can only intersect by containement, there exist one set that contains all places.
\item PN token is identified as execution flow and dependency flow that the model try to capture.
\item async tasks run \emph{inside} the Zone.
\item Zoned results may be used \emph{multiple} times.
\end{itemize}

\section{Terminology}

Borrowing abstraction from the petri nets, I define the terms:

\begin{itemize}
\item Token, Zoned Token : Input, output, execution flow that crosses the zones
\item Task, Zoned Task : Code to be executed, with possible input and output (resp. zoned code)
\item Token Hooks or crossing hooks : Code defined by zone, triggered when a token crosses the zone boundary.
\end{itemize}

\section{Properties}

Now that the zone define an operation mode matching (a)synchronous uniformity, I propose the set of essential properties.

\begin{itemize}
\item Zone Values : as in zone prototype
\item Around hooks : function from task to task, Union inherited
\item Crossing Hooks : catch crossing tokens
\end{itemize}

\section{Binding requirements}

In order to make the Zone easy and clear to use, it needs a binding procedure specification.

\begin{itemize}
\item general three-phase binding
\item asynchronous binding
\item synchronous binding
\item internal binding
\end{itemize}

\section{Applications}

At this point, you may wonder if the generalization has gone too far. No error handling, no inherited before hooks, ... I show here how to combine the Zone properties to build up expected features.

\begin{itemize}
\item Contextual and asynchronous error Handling
\item Long stack traces or one Zone per task
\item Asynchronous Dependency tracing
\item Simulated inheritance for arbitrary custom properties (like javascript before hook)
\end{itemize}