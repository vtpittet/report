\begin{abstract}

This research project targets the consistency concerns one meets when he write asynchronous programs. It offers overview and analysis of cases where our synchronous intuition (or \emph{mental model}) of program execution tricks us in assuming wrong hypothesis. It invites you to take some time to stop saying yourself ``I know how it works'' and try to explain yourself ``How does it work''.
In a sense, it tries to show the best answer to ``Why doesn't this code work ?'' may be ``Why did you think this would have worked ?''.
This allows to state clear limits on what \emph{approach} is possible or not and provide a way to think your problems that match the programming model you use
``What you want to do is simply impossible because it doesn't work that way. However the effect you want to archeive is possible, but require you to adapt your way of thinking to match the model you are using and not the model you \emph{think} you are using.''
Note that this work is not about concurrency at all. Some part may address this concern, discussing thread safety but no solution proposed here solves concurrency problems. At most, it makes the code user aware of a risk of race conditions.


\end{abstract}